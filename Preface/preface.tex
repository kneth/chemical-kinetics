%%%%%%%%%%%%%%%%%%%%%%%%%%%%%%%%%%%%%%%%%%%%%%%%%%%%%%%%%%%%%%%%%%%%%%%%%%%%%%%
%%
%% (C) Copyright 1996, 2020-2021 Kenneth Geisshirt <geisshirt@gmail.com>
%%
%% Licensed under Creative Commons Attribution-ShareAlike 4.0 International.
%% You may obtain a copy of the license at
%% https://creativecommons.org/licenses/by-sa/4.0/
%%
%%%%%%%%%%%%%%%%%%%%%%%%%%%%%%%%%%%%%%%%%%%%%%%%%%%%%%%%%%%%%%%%%%%%%%%%%%%%%%%

\chapter*{Preface}

Chemical kinetics is regarded as part of physical chemistry, and therefore it is part of any introductory course in physical chemistry.

Despite the name, thermodynamics is a science of equilibrium \ie the systems studied do not change with time. Chemical kinetics is basically the opposite -- here we study the change of the concentration of a species during a chemical reaction.

The present text is about chemical kinetics -- as hinted in the title. The intended reader is an undergraduate student in chemistry. Chemical kinetics - as physical chemistry in general - draws a lot from mathematics and the essential mathematics used in this text is explained or outlined in an appendix.

The motivation for writing this text is, that the author has not yet found a good introductory textbook on chemical kinetics. Any contemporary textbook on physical chemistry has a chapter or two, but often they are either too badly written or too simple. There exist some textbooks covering only chemical kinetics, but they are for either advanced undergraduate or graduate students.
