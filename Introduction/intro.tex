\chapter{Introduction and basic definitions}
\label{chap:intro}

Let us begin by defining and delimiting our term ``chemical
kinetics''. As already mentioned in the preface, chemical kinetics is
not about equilibrium, but the dynamics of chemical reactions.

We can choose two distinct approaches towards chemical kinetics, namely
the microscopic or the macroscopic one. The difference is the size of
the system studied. In the microscopic approach we will study one
molecule reacting with another molecule, while the macroscopic
approach is a more phenomenological one. Here we study many molecules,
maybe 1 mole (remember that 1 mole is $6.022\times 10^{23}$
molecules).

The present text approach chemical kinetics from the macroscopic
viewpoint.

The dynamics of the macroscopic viewpoint is the change in
concentration. In other words, we should imaging that we are
monitoring the concentration of a given species as a function of
time. For example imaging the reaction $\species{I_2} + \species{I} \rightarrow \species{I_3}$. In this
case we could measure the concentration of triodide \ie
$[\species{I_3}](t)$ where the $t$ indicates that the concentration is
measured as a function of time.

I have to come with a warning before we begin our journey into the
land of chemical kinetics. Often we will not use real chemical
species, but only abstract ones like $A$ and $B$. The convention is
that capital letters are always chemical species. In the beginning this
may seem strange, but once learned, we can apply our results to many
(real) reactions.

\section{Velocity of a reaction}

In order to do a more formalized (\ie mathematical) analysis of
chemical reactions, it is not enough just to measure the concentration
of the species involved. A more convenient approach is to begin the
analysis by discussing the velocity of the reaction.

The most general form we can write up a chemical reaction as

\[
  aA + bB + \cdots + xX + yY + \cdots \rightarrow 0
\]

where $A$, $B$ are reactants, $a$, $b$ their stoichiometric
coefficients, $X$, $Y$ the products, and $x$, $y$ their stoichiometric
coefficients. Remember, our convention is that the stoichiometric
coefficients for reactants are negative, and positive for products.

We will \textit{define} the velocity of this reaction as

\begin{equation}
  \label{eq:velocity}
  v = \frac{1}{a}\dt{[A]} = \frac{1}{b}\dt{[B]} = \frac{1}{x}\dt{[X]} = \frac{1}{y}\dt{[Y]}
\end{equation}

\begin{example}
  We will illustrate the concept of velocity by looking at the decomposition of hydrogen peroxide \ie

  \[
      2\species{H_2O_2} \rightarrow 2\species{H_2O} + \species{O_2}
  \]

  We have three species ($\species{H_20_2}$, $\species{H_2O}$, and $\species{O_2}$), and
  the stoichiometric coefficients are —2, 2 and 1, respectively. The
  velocity is therefore

  \[
    v = -\frac{1}{2}\dt{[\species{H_2O_2}]} = \frac{1}{2}\dt{[\species{H_2O}]} = \dt{[\species{O_2}]}
  \]
\end{example}

So far we have not said anything about the velocity. If we are able to
give an expression for $v$, we would have a differential equation. By
solving this differential equation we would have the concentration as
function of time (which is comparable to experiments). It is important
to stress that the expression for the velocity is an experimental
fact.

Often we call the expression for the velocity of the reaction for the
\textit{rate law}.

\section{Elementary reactions}
\label{sec:elementaryReactions}

All chemical reactions can be divided into two classes, elementary and
non-elementary reactions. The elementary reactions are the ones that
happen in Nature (usually our laboratory). Any non-elementary reaction
is the sum of a number of elementary reactions.

Elementary reactions usually only have one or two reactants. But as we
will see later the reaction $\species{H_2} + \species{Br_2} \rightarrow 2\species{HBr}$ is not an
elementary reaction as one might first think. Enzyme reactions are
never written down in terms of elementary reactions while gas phase
reactions usually are.

Elementary reactions have one nice feature; we can - almost
automatically - write down an expression for the velocity of the
reaction. Or we could say it the opposite way: if the reaction follows
a certain expression for the velocity (an expression we will discuss
below) we know that it is an elementary reaction.

Let us use a generalized reaction \ie

\[
  aA + bB + \cdots \rightarrow xX + yY \cdots
\]

where the stoichiometric coefficients for both the reactants and the
products are positive!

For an elementary reaction the velocity is

\begin{equation}
  \label{eq:elementaryReaction}
  v = k[A]^a[B]^B\cdots
\end{equation}

The constant $k$ is a proportionality constant which we call the
\textit{rate constant}.

\begin{example}
  We continue our example from previous section \ie the decomposition of hydrogen peroxide:

  \begin{equation}
    \label{eq:decompH2O2}
    2\species{H_2O_2} \rightarrow 2\species{H_2O} + \species{O_2}
  \end{equation}

  There is only one reactant, $\species{H_2O_2}$, and the velocity is

  \begin{equation}
    \label{eq:decompH2O2velocity}
    v = k\cdot [\species{H_2O_2}]^2
  \end{equation}
\end{example}

In the example we can see one of the bad habits used in chemical
kinetics. The concentration of hydrogen peroxide is written as
$[\species{H_2O_2}]$, but this notation does not emphasize that the
concentration charges with time. The correct way of writing the
concentration of hydrogen peroxide is $[\species{H_2O_2}](t)$, but we will
never do that.

Some chemical kineticists call reactions where the velocity is written
as in equation \ref{eq:elementaryReaction} kinetics consistent with
the \textit{law of mass action}. The reason for this is that the
expression for the velocity reminds us of the one used in the
discussion of chemical equilibrium. Of course nobody uses so long a
term everyday - one just says: mass action kinetics.

Ending the discussion of the concept of elementary reactions we will
introduce the order of a reaction. The term ``order'' is only
meaningful when we are talking about mass action kinetics. The order
of a reaction is defined as the sum of the stoichiometric coefficients
of the reactants (or more precisely, the sum of the exponents in the
rate law). In the example of the decomposition of hydrogen peroxide,
the order is 2.
